\documentclass[12pt,english]{article}
\usepackage[a4paper,bindingoffset=0.2in,%
            left=1in,right=1in,top=1in,bottom=1in,%
            footskip=.25in]{geometry}
\usepackage{blindtext}
\usepackage{titling}
\usepackage{amssymb}
\usepackage{amsmath}
\usepackage{listings}
\usepackage{lettrine} 
\usepackage{tikz}  
\usepackage{color} 
\usepackage{verbatim}
\usepackage{pgfplots}
\usepgfplotslibrary{external}
\pgfplotsset{width=10cm,compat=1.9}
%================================
\begin{document}
\newgeometry{left=0.8in,right=0.8in,top=1in,bottom=1in}
\begin{center}
    \Large
    \textbf{Homework 7 G}\\
    \small
    \today\\
    \large
    Jose Carlos Munoz
\end{center}%===============================
19.a)\\

19.b)\\
19.f)\\
24)\par
The possible pairs are the sets \{P1,P2\},\{P1,P3\},\{P1,P4\},\{P2,P3\},\{P2,P4\},\{P3,P4\}.\par
Based on the ideal similarity matrix we get the set x= \{1,0,0,0,0,1\}.\par
In the  similarity matrix the we get the set y = \{0.8,0.65,0.55,0.7,0.6,0.9\}.\par
The $\sigma_x$ is $0.5164$ and $\sigma_y$ is 0.1304.The cov(x,y) = 0.06. \par
To find the correlation its $\frac{cov(x,y)}{\sigma_x * \sigma_y}$.\par
So the correlation value is 0.08910.\\
25)\par
To find the F(i,j) value we first find the R(i,j) and P(i,j).R(i,j) is equal to $\frac{n_{ij}}{n_i}$. Where $n_{ij}$ is the amount of class a in the cluster and $n_i$ is how many class values over all.P(i,j) is equal to $\frac{n_{ij}}{n_j}$. Where $n_{ij}$ is the amount of class a in the cluster and $n_i$ is how many values in the cluster.\\
F(i,j) is equal to $2 * R(i,j) * \frac{P(i,j)}{P(i,j) + R(i,j)}$
For Cluster 1\par
For Class A\par
R(A,1) = $\frac{3}{3}$ = 1, P(A,1) = $\frac{3}{8}$\par
F(A,1) = $2 * 1 \frac{1}{1 + 3/8}$ = 0.55\par
For Class B\par
R(B,1) = $\frac{5}{5}$ = 1, P(B,1) = $\frac{5}{8}$\par
F(B,1) = $2 * 1 \frac{1}{1 + 5/8}$ = 0.77\\
For Cluster 2\par
For Class A\par
R(A,2) = $\frac{2}{3}$, P(A,2) = $\frac{2}{4}$, F(A,2) = 0.57\par
For Class B\par                            
R(B,2) = $\frac{2}{5}$, P(B,2) = $\frac{2}{4}$, F(B,2) = 0.44\\
For Cluster 3\par                          
For Class A\par                            
R(A,3) = $\frac{1}{3}$, P(A,3) = $\frac{1}{4}$, F(A,3) = 0.29\par
For Class B\par                            
R(B,3) = $\frac{3}{5}$, P(B,3) = $\frac{3}{4}$, F(B,3) = 0.67\\
For Cluster 4\par                          
For Class A\par                            
R(A,4) = $\frac{2}{3}$, P(A,4) = $\frac{2}{2}$, F(A,4) = 0.80\par
For Class B\par                            
R(B,4) = $\frac{0}{5}$, P(B,4) = $\frac{0}{4}$, F(B,4) = 0.00\\
For Cluster 5\par                          
For Class A\par                            
R(A,5) = $\frac{0}{3}$, P(A,5) = $\frac{0}{2}$, F(A,5) = 0.00\par
For Class B\par                            
R(B,5) = $\frac{2}{5}$, P(B,5) = $\frac{2}{2}$, F(B,5) = 0.57\\
For Cluster 6\par                          
For Class A\par                            
R(A,6) = $\frac{1}{3}$, P(A,6) = $\frac{1}{2}$, F(A,6) = 0.40\par
For Class B\par                            
R(B,6) = $\frac{1}{5}$, P(B,6) = $\frac{1}{2}$, F(B,6) = 0.29\\
For Cluster 7\par                          
For Class A\par                            
R(A,7) = $\frac{0}{3}$, P(A,7) = $\frac{0}{2}$, F(A,7) = 0.00\par
For Class B\par                            
R(B,7) = $\frac{2}{5}$, P(B,7) = $\frac{2}{2}$, F(B,7) = 0.57\par
For Overall Clustering we have to use the Max F(A) and F(B) values which are 0.8 and 0.77 respectively.\par
The value is $\frac{3}{8} * 0.8 +\frac{5}{8} * 0.77$ which is 0.78\\\
26)\\
\end{document}
