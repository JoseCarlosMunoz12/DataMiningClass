\documentclass[12pt,english]{article}
\usepackage[a4paper,bindingoffset=0.2in,%
            left=1in,right=1in,top=1in,bottom=1in,%
            footskip=.25in]{geometry}
\usepackage{blindtext}
\usepackage{titling}
\usepackage{amssymb}
\usepackage{amsmath}
\usepackage{listings}
\usepackage{lettrine} 
\usepackage{tikz}  
\usepackage{color} 
\usepackage{verbatim}
 \usetikzlibrary{shapes, arrows, calc, arrows.meta, fit, positioning} % these are the parameters passed to the library to create the node graphs  
\tikzset{  
    -Latex,auto,node distance =0.4 cm and 1.0 cm, thick,% node distance is the distance between one node to other, where 1.5cm is the length of the edge between the nodes  
    state/.style ={ellipse, draw, minimum width = 0.3 cm}, % the minimum width is the width of the ellipse, which is the size of the shape of vertex in the node graph  
    point/.style = {circle, draw, inner sep=0.18cm, fill, node contents={}},  
    bidirected/.style={Latex-Latex,dashed}, % it is the edge having two directions  
  treenode/.style = {align=center, inner sep=0pt, text centered,
    font=\sffamily},
    el/.style = {inner sep=2.5pt, align=right, sloped},
    arn_r/.style = {treenode, circle, red, draw=red},
    arn_x/.style = {treenode, rectangle, draw=black},
    minimum width=2.5em, minimum height=1.5em}% arbre rouge noir, nil
\setlength{\parskip}{12pt}
\title{Home Work 3 Undergraduate}
\date{\today}
\author{Jose Carlos Munoz}
%================================
\begin{document}
\newgeometry{left=0.8in,right=0.8in,top=1in,bottom=1in}
\begin{center}
    \Large
    \textbf{Homework 3 G}\\
    \small
    \today\\
    \large
    Jose Carlos Munoz
\end{center}%===============================
To find the Z score we calculate it with the following equation
\begin{equation}\tag{1}\label{eq:1}
Z = \frac{p_a - p_b}{\sqrt{\frac{2p(1-p)}{N}}}
\end{equation}
$p_a$ represents the proportion of Classification A and $p_b$ for Classificiation B. p is the average proportion of both $p_a$ and $p_b$\\
Classification A is significantly better that Classification B if the Z value is greater than 1.96. And Classification B is significantly better if the Z score is below -1.96. Both are a draw if the Z value is between both of them.\\
The resulting comparison are shown in the following table which shows the wins - loss- and draws for each comparison.\\
Calculations where done in the python file
\begin{equation}
\begin{array}{c|ccc}
\mbox{-}& \mbox{DT} & \mbox{GD} & \mbox{V}\\
\hline
\mbox{DT} & \left[ 0,0,23 \right]& \left[ 10,2,11\right]& \left[ 2,6,15 \right]\\
\mbox{GD} & \left[ 2,10,11\right]& \left[ 0,0,23 \right]& \left[ 0,8,15 \right]\\
\mbox{V}  & \left[ 6,2,15 \right]& \left[ 8,0,15 \right]& \left[ 0,0,23 \right]
\end{array}
\end{equation}
\end{document}
