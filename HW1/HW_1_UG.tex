\documentclass[12pt,english]{article}
\usepackage[a4paper,bindingoffset=0.2in,%
            left=1in,right=1in,top=1in,bottom=1in,%
            footskip=.25in]{geometry}
\usepackage{blindtext}
\usepackage{titling}
\usepackage{amssymb}
\usepackage{amsmath}
\usepackage{listings}
\usepackage{lettrine} 
\usepackage{tikz}  
\usepackage{color} 
 \usetikzlibrary{shapes, arrows, calc, arrows.meta, fit, positioning} % these are the parameters passed to the library to create the node graphs  
\tikzset{  
    -Latex,auto,node distance =0.6 cm and 1.3 cm, thick,% node distance is the distance between one node to other, where 1.5cm is the length of the edge between the nodes  
    state/.style ={ellipse, draw, minimum width = 0.9 cm}, % the minimum width is the width of the ellipse, which is the size of the shape of vertex in the node graph  
    point/.style = {circle, draw, inner sep=0.18cm, fill, node contents={}},  
    bidirected/.style={Latex-Latex,dashed}, % it is the edge having two directions  
    el/.style = {inner sep=2.5pt, align=right, sloped}  
}  
\setlength{\parskip}{12pt}
\title{Home Work 1 Undergraduate}
\date{\today}
\author{Jose Carlos Munoz}
%================================
\begin{document}
\newgeometry{left=0.8in,right=0.8in,top=1in,bottom=1in}
\begin{center}
    \Large
    \textbf{Homework 1 Undergraduate}\\
    \small
    \today\\
    \large
    Jose Carlos Munoz
\end{center}
Part1)\\
1)\\
It is given that the Data set is set up as D.\\
\begin{equation*}
D =
\begin{bmatrix}19 & 12 \\22 &  6 \\6  &  9 \\3 &  15 \\2 &  13 \\20 &  5 \\\end{bmatrix}
\end{equation*}
To center it, $\mu$ must be found for both attributs and subtracted from D.
\begin{equation*}
\begin{split}
\mu_{1} &= (19+22+6+3+2+10) / 60\\
 &= 12 \\
\mu_{2} &= (12+6+9+15+13+5) / 60\\
 &= 10
\end{split}
\end{equation*}
Once found, $\bar{D}= D -\bold{1}*\mu$
\begin{equation*}
\begin{split}
\bar{D}&=
\begin{bmatrix}19 & 12 \\22 &  6 \\ 6  &  9 \\ 3 &  15 \\ 2 &  13 \\20 &  5 \end{bmatrix}
-
\begin{bmatrix}12 & 10 \\12 & 10 \\12 & 10 \\12 & 10 \\12 & 10 \\12 & 10 \end{bmatrix}\\
\bar{D}&=
\begin{bmatrix}7 &  2 \\10 & -4 \\-6 & -1 \\-9 &  5 \\-10 &  3 \\8 & -5 \end{bmatrix}
\end{split}
\end{equation*}
\newpage
2)\\
Finding the Covariance of a centerd matrix is $C_{x} = \frac{1}{n-1} * \bar{D}^{T}*\bar{D}$.Where n is the number of objects in the matrix. 
\begin{equation*}
\begin{split}
C_{x}&= \frac{1}{6-1}
*
\begin{bmatrix}7 & 10 & -6& -9 & -10 & 8\\2 & -4 & -1& 5 & 3 &-5\end{bmatrix}
*
\begin{bmatrix}7 &  2 \\10 & -4 \\-6 & -1 \\-9 &  5 \\-10 &  3 \\8 & -5 \end{bmatrix}\\
C_{x} &= \frac{1}{5}
*
\begin{bmatrix}430 & -135 \\-135 & 80\end{bmatrix}\\
C_{x} &= 
\begin{bmatrix}86 & -27 \\-27 & 16\end{bmatrix}
\end{split}
\end{equation*}
%In the middle if y
3) To find the characteristics poolynomials we have to take the determinate of the Covariance minus the eigen values. 
$det(C_{x} -\lambda *1)$
\begin{equation*}
\begin{split}
det(
\begin{bmatrix}86 & -27 \\-27 & 16\end{bmatrix}
- 
\begin{bmatrix}\lambda & 0 \\0 & \lambda\end{bmatrix}) = 0\\
det(
\begin{bmatrix}86 - \lambda & -27 \\-27 & 16 - \lambda
\end{bmatrix}) = 0\\
(86-\lambda)(16-\lambda) - (-27)(-27) = 0\\
1376 -102\lambda + \lambda^{2} - 729 = 0\\
\lambda^{2} -102\lambda + 647 = 0
\end{split}
\end{equation*}
From the polynomial $\lambda^{2} -102\lambda + 647 $, the components a, b and c are 1,-102, and 647 respectively. To find the Eigen Values, we find the root of the polynomials. Which can be done with the quadratic equation. Giving us 2  roots; $\lambda_{1} = 95.204072$ and $\lambda_{2} = 6.795928$.
\newpage
4)The principal component matrix is made up from the Eigen Vectors of $C_{x}$. Using R  the Pricipal Component matix is
\begin{equation*}
\begin{bmatrix}-0.9465153 & -0.3226591 \\0.3226591 & -0.9465153
\end{bmatrix}
\end{equation*}
5)The first principal component $p_{1}$ is the chosen component devided by the sum of all the components. So the percentage is $0.9333733$.\\
6)THe PCA transformation is found by $\bold{Y} = \bold{P}^{-1} * \bold{X}$.
\begin{equation*}
\begin{split}
\bold{P}^{-1} &=\begin{bmatrix}-0.9465153 & 0.3226591 \\-0.3226591 & -0.9465153\end{bmatrix}\\
\bold{Y} &=\begin{bmatrix}19 & 12 \\22 &  6 \\ 6  &  9 \\ 3 &  15 \\ 2 &  13 \\20 &  5 \end{bmatrix}
 *
\begin{bmatrix}-0.9465153 & 0.3226591 \\-0.3226591 & -0.9465153\end{bmatrix}\\
&=\begin{bmatrix}-14.111881 &-17.48871 \\ -18.887381 &-12.77759 \\ -2.775160 &-10.45459 \\ 2.000340 &-15.16571 \\ 2.301537 & -12.95002 \\ -17.317010 &-11.18576 \end{bmatrix}
\end{split}
\end{equation*}
7)\\
i)N here is the number of Attributes in the data set. THe percetange of variance state how correlated the variables are\\
ii)The results of i can be intepret that PCA is not helpful at all. Since we want to reduce the number of dimensions and see the best one. We started with N dimensions and still remain with N dimensions. The case for ii is much more favorable. since only one principal component is needed and show a possible correlation, we can reduce from N dimensions to just 1.
%===============================
\end{document}
