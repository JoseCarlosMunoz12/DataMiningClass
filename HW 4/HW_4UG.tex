\documentclass[12pt,english]{article}
\usepackage[a4paper,bindingoffset=0.2in,%
            left=1in,right=1in,top=1in,bottom=1in,%
            footskip=.25in]{geometry}
\usepackage{blindtext}
\usepackage{titling}
\usepackage{amssymb}
\usepackage{amsmath}
\usepackage{listings}
\usepackage{lettrine} 
\usepackage{tikz}  
\usepackage{color} 
\usepackage{verbatim}
 \usetikzlibrary{shapes, arrows, calc, arrows.meta, fit, positioning} % these are the parameters passed to the library to create the node graphs  
\tikzset{  
    -Latex,auto,node distance =0.4 cm and 1.0 cm, thick,% node distance is the distance between one node to other, where 1.5cm is the length of the edge between the nodes  
    state/.style ={ellipse, draw, minimum width = 0.3 cm}, % the minimum width is the width of the ellipse, which is the size of the shape of vertex in the node graph  
    point/.style = {circle, draw, inner sep=0.18cm, fill, node contents={}},  
    bidirected/.style={Latex-Latex,dashed}, % it is the edge having two directions  
  treenode/.style = {align=center, inner sep=0pt, text centered,
    font=\sffamily},
    el/.style = {inner sep=2.5pt, align=right, sloped},
    arn_r/.style = {treenode, circle, red, draw=red},
    arn_x/.style = {treenode, rectangle, draw=black},
    minimum width=2.5em, minimum height=1.5em}% arbre rouge noir, nil
\setlength{\parskip}{12pt}
\title{Home Work 3 Undergraduate}
\date{\today}
\author{Jose Carlos Munoz}
%================================
\begin{document}
\newgeometry{left=0.8in,right=0.8in,top=1in,bottom=1in}
\begin{center}
    \Large
    \textbf{Homework 4 UG}\\
    \small
    \today\\
    \large
    Jose Carlos Munoz
\end{center}%===============================
3.10)
To find the cost function of both of the Decision tree we use this formula.
\begin{equation}
F_x(n) = \mbox{Nodes} * \log_{2}m + \mbox{Leafs} * \left [ \log_{2}k \right] + \mbox{Errors} *\log_{2} n\\
\end{equation}
Where m is the number of Attributes, k is the number of classes and n is the unkown Sample size. Which are 16, 3 and n respectivley\\
Now we find the cost function for decision Tree A, where Nodes = 2, Leafs = 3, 7 errors
\begin{equation}\tag{a}\label{eq:a}
\begin{split}
F_a(n) &= 2 * \log_{2}16 + 3 * \left [ \log_{2}3 \right] + 7 *\log_{2} n\\
F_a(n) &= 2 * 4 + 3 * 2 + 7 *\log_{2} n\\
F_a(n) &= 8 + 6 + 7 *\log_{2} n\\
F_a(n) &= 14 + 7 *\log_{2} n\\
\end{split}
\end{equation}
Now we find the cost function for decision Tree b, where Nodes = 2, Leafs = 3, and 4 errors.
\begin{equation}\tag{b}\label{eq:b}
\begin{split}
F_b(n) &= 4 * \log_{2}16 + 5 * \left [ \log_{2}3 \right] + 4 *\log_{2} n\\
F_b(n) &= 4 * 4 + 5 * 2 + 4 *\log_{2} n\\
F_b(n) &= 16 + 10 + 4 *\log_{2} n\\
F_b(n) &= 26 + 4 *\log_{2} n\\
\end{split}
\end{equation}
4.6)a\\
\begin{equation}\tag{1}\label{eq:1}
\begin{split}
P(\mbox{S}\vert \mbox{UG}) &= .15 \\
P(S\vert G) &= .23\\
P(G) &= .2\\
P(UG) &= .8
\end{split}
\end{equation}
These are the known probabilites.\\
 From this we can find $P(\mbox{G}\vert \mbox{S})$.\\
Because of Bayes Theroem  $P(\mbox{G}\vert \mbox{S})$ is the same as the following
\begin{equation}
P(\mbox{G}\vert \mbox{S}) = \frac{P(\mbox{S} \vert \mbox{G}) * P(\mbox{G})}{P(\mbox{S})}
\end{equation}
$P(\mbox{S})$ can be found as
\begin{equation}
\begin{split}
P(\mbox{S}) &=  P(\mbox{S} \vert \mbox{G}) * P(\mbox{G}) + P(\mbox{S} \vert \mbox{UG}) * P(\mbox{UG})\\
P(\mbox{S}) &= .23 *.2 + .15 * .8\\
P(\mbox{S}) &=.166
\end{split}
\end{equation}
Therefore
\begin{equation}
\begin{split}
P(\mbox{G}\vert \mbox{S}) &= \frac{.23 * .2}{.166}\\
P(\mbox{G}\vert \mbox{S}) &= .277\\
\end{split}
\end{equation}
 So the probabilty that a smoker is a graduate student is .277\par
 4.6)c\\
The probability that a smoker is a graduated student  can be written as  $P(\mbox{UG}\vert \mbox{S})$.\\
\begin{equation}
\begin{split}
P(\mbox{UG}\vert \mbox{S}) &=  \frac{P(\mbox{S} \vert \mbox{UG}) * P(\mbox{UG})}{P(\mbox{S})}\\
P(\mbox{UG}\vert \mbox{S}) &=\frac{.23 * .8}{.277}\\
P(\mbox{UG}\vert \mbox{S}) &=.857\\
\end{split}
\end{equation}
 So the probabilty that a smoker is an undergrad is  .857.\\
 Since $P(\mbox{UG}\vert \mbox{S}) > P(\mbox{G}\vert \mbox{S})$ 
 we can conclude we have a higher chance of finding an undergrad that is a smoker\par
 4.6)d\\
\begin{equation}
\begin{split}
P(\mbox{D}\vert \mbox{UG}) &= .1\\
P(\mbox{D}\vert \mbox{G}) &= .3\\
P(\mbox{D}) &= P(\mbox{D}\vert \mbox{UG}) * P(\mbox{UG})+ P(\mbox{D}\vert \mbox{G}) *  P(\mbox{G})\\
P(\mbox{D}) &= 0.1 * .8 + .2 * .3\\
P(\mbox{D}) &= .14\\
P(\mbox{D,S} \vert \mbox{G}) &=P(\mbox{D}\vert \mbox{G}) *P(\mbox{S}\vert \mbox{G})\\
P(\mbox{D,S} \vert \mbox{G}) &= .3 * .23\\
P(\mbox{D,S} \vert \mbox{G}) &= .069\\
P(\mbox{D,S} \vert \mbox{UG}) &=P(\mbox{D}\vert \mbox{UG}) *P(\mbox{S}\vert \mbox{UG})\\
P(\mbox{D,S} \vert \mbox{UG}) &= .1 * .15\\
P(\mbox{D,S} \vert \mbox{UG}) &=0.015\\
P(\mbox{D,S}) &= Q
\end{split}
\end{equation}
These are the known probabilites. Since we don't know what $P(\mbox{D,S})$ is, we set it as a constant Q\\
Now we can find the values for $P(\mbox{G}\vert \mbox{D,S})$ and $P(\mbox{UG}\vert \mbox{D,S})$\\
\begin{equation}
\begin{split}
P(\mbox{UG}\vert \mbox{D,S}) &=  \frac{P(\mbox{D,S} \vert \mbox{UG}) * P(\mbox{UG})}{P(\mbox{D,S})}\\
P(\mbox{UG}\vert \mbox{D,S}) &=\frac{.015 * .8}{Q}\\
P(\mbox{UG}\vert \mbox{D,S}) &=\frac{.012}{Q}\\
P(\mbox{G}\vert \mbox{D,S}) &=  \frac{P(\mbox{D,S} \vert \mbox{UG}) * P(\mbox{UG})}{P(\mbox{D,S})}\\
P(\mbox{G}\vert \mbox{D,S}) &=\frac{.069 *.2}{Q}\\
P(\mbox{G}\vert \mbox{D,S}) &=\frac{.0139}{Q}\\
\end{split}
\end{equation}
From these results we can conclude that the chance that we find a graduate that lives in a dorm and is a smoker is  is higher than the chance that we find an undergraduate that lives in a dorm and is a smoker.\par
4.7)a\\
\begin{equation}
\begin{split}
P(\mbox{A=0} \vert +) &= \frac{2}{5} = .4\\
P(\mbox{A=0} \vert -) &= \frac{3}{5} = .6\\
P(\mbox{A=1} \vert +) &= \frac{3}{5} = .6\\
P(\mbox{A=1} \vert -) &= \frac{2}{5} = .4\\
P(\mbox{B=0} \vert +) &= \frac{4}{5} = .8\\
P(\mbox{B=0} \vert -) &= \frac{3}{5} = .6\\
P(\mbox{B=1} \vert +) &= \frac{1}{5} = .2\\
P(\mbox{B=1} \vert -) &= \frac{2}{5} = .4\\
P(\mbox{C=0} \vert +) &= \frac{3}{5} = .6\\
P(\mbox{C=0} \vert -) &= \frac{0}{5} = 0\\
P(\mbox{C=1} \vert +) &= \frac{2}{5} = .4\\
P(\mbox{C=1} \vert -) &= \frac{5}{5} = .1
\end{split}
\end{equation}
\par
4.7)b\\
we are task to find $P(\mbox{A=0,B=1,C=0} \vert \mbox{+})$. Using the Bayes Therm we canfind the value as
\begin{equation}
\begin{split}
P(\mbox{+} \vert \mbox{A=0,B=1,C=0}) &= \frac{P(\mbox{A=0,B=1,C=0} \vert \mbox{+}) * P(\mbox{+})}{P(\mbox{A=0,B=1,C=0})}\\
P(\mbox{+} \vert \mbox{A=0,B=1,C=0}) &= \frac{P(\mbox{A=0} \vert \mbox{+}) * P(\mbox{B=1} \vert \mbox{+}) *P(\mbox{C=0} \vert \mbox{+}) * P(\mbox{+})}{P(\mbox{A=0,B=1,C=0})}\\
P(\mbox{+} \vert \mbox{A=0,B=1,C=0}) &= \frac{.4 * .2 *.6 * .5}{P(\mbox{A=0,B=1,C=0})}\\
P(\mbox{+} \vert \mbox{A=0,B=1,C=0}) &= \frac{0.024}{P(\mbox{A=0,B=1,C=0})}\\
\end{split}
\end{equation}
\begin{equation}
\begin{split}
P(\mbox{-} \vert \mbox{A=0,B=1,C=0}) &= \frac{P(\mbox{A=0,B=1,C=0} \vert \mbox{-}) * P(\mbox{-})}{P(\mbox{A=0,B=1,C=0})}\\
P(\mbox{-} \vert \mbox{A=0,B=1,C=0}) &= \frac{P(\mbox{A=0} \vert \mbox{-}) * P(\mbox{B=1} \vert \mbox{-}) *P(\mbox{C=0} \vert \mbox{-}) * P(\mbox{-})}{P(\mbox{A=0,B=1,C=0})}\\
P(\mbox{-} \vert \mbox{A=0,B=1,C=0}) &= \frac{.6 * .4 *0 * .5}{P(\mbox{A=0,B=1,C=0})}\\
P(\mbox{-} \vert \mbox{A=0,B=1,C=0}) &= 0\\
\end{split}
\end{equation}
\par
From these results we canconlude that the class label for (A=0, B=1, C=0) will be Class +.\par
4.7)c\\
We will be looking at the conditional probabilites for the them all over again with the m-estimate. When m=4 and p = 1/2;
to find the new Conditonal probabilites we use this equation
\begin{equation}
\frac{ n_c + m * p}{n + m}
\end{equation}
so now the The conditional probablities will be
\begin{equation}
\begin{split}
P(\mbox{A=0} \vert +) &= \frac{2 + 2}{5 + 4} = \frac{4}{9}\\
P(\mbox{A=0} \vert -) &= \frac{3 + 2}{5 + 4} = \frac{5}{9}\\
P(\mbox{A=1} \vert +) &= \frac{3 + 2}{5 + 4} = \frac{5}{9}\\
P(\mbox{A=1} \vert -) &= \frac{2 + 2}{5 + 4} = \frac{4}{9}\\
P(\mbox{B=0} \vert +) &= \frac{4 + 2}{5 + 4} = \frac{6}{9}\\
P(\mbox{B=0} \vert -) &= \frac{3 + 2}{5 + 4} = \frac{5}{9}\\
P(\mbox{B=1} \vert +) &= \frac{1 + 2}{5 + 4} = \frac{3}{9}\\
P(\mbox{B=1} \vert -) &= \frac{2 + 2}{5 + 4} = \frac{4}{9}\\
P(\mbox{C=0} \vert +) &= \frac{3 + 2}{5 + 4} = \frac{5}{9}\\
P(\mbox{C=0} \vert -) &= \frac{0 + 2}{5 + 4} = \frac{2}{9}\\
P(\mbox{C=1} \vert +) &= \frac{2 + 2}{5 + 4} = \frac{4}{9}\\
P(\mbox{C=1} \vert -) &= \frac{5 + 2}{5 + 4} = \frac{7}{9}
\end{split}
\end{equation}
\par
4.7)d\\
we repeat b) but with the m-estimate conditional probabilities
\begin{equation}
\begin{split}
P(\mbox{+} \vert \mbox{A=0,B=1,C=0}) &= \frac{P(\mbox{A=0,B=1,C=0} \vert \mbox{+}) * P(\mbox{+})}{P(\mbox{A=0,B=1,C=0})}\\
P(\mbox{+} \vert \mbox{A=0,B=1,C=0}) &= \frac{P(\mbox{A=0} \vert \mbox{+}) * P(\mbox{B=1} \vert \mbox{+}) *P(\mbox{C=0} \vert \mbox{+}) * P(\mbox{+})}{P(\mbox{A=0,B=1,C=0})}\\
P(\mbox{+} \vert \mbox{A=0,B=1,C=0}) &= \frac{\frac{4}{9} * \frac{3}{9} *\frac{5}{9} * .5}{P(\mbox{A=0,B=1,C=0})}\\
P(\mbox{+} \vert \mbox{A=0,B=1,C=0}) &= \frac{0.0142}{P(\mbox{A=0,B=1,C=0})}\\
\end{split}
\end{equation}
\begin{equation}
\begin{split}
P(\mbox{-} \vert \mbox{A=0,B=1,C=0}) &= \frac{P(\mbox{A=0,B=1,C=0} \vert \mbox{-}) * P(\mbox{-})}{P(\mbox{A=0,B=1,C=0})}\\
P(\mbox{-} \vert \mbox{A=0,B=1,C=0}) &= \frac{P(\mbox{A=0} \vert \mbox{-}) * P(\mbox{B=1} \vert \mbox{-}) *P(\mbox{C=0} \vert \mbox{-}) * P(\mbox{-})}{P(\mbox{A=0,B=1,C=0})}\\
P(\mbox{-} \vert \mbox{A=0,B=1,C=0}) &= \frac{\frac{5}{9} * \frac{4}{9} *\frac{2}{9} .5}{P(\mbox{A=0,B=1,C=0})}\\
P(\mbox{-} \vert \mbox{A=0,B=1,C=0}) &= \frac{0.0274}{P(\mbox{A=0,B=1,C=0})}\\
\end{split}
\end{equation}
From these resutl we canconlude that the class label for (A=0,B=1,C=0) is class +\\
\par
4.7)e\\
The better method would be the m-estimate becuase we do not want our entire expression to be zero
\end{document}
