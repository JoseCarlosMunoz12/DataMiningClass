
\documentclass[12pt,english]{article}
\usepackage[a4paper,bindingoffset=0.2in,%
            left=1in,right=1in,top=1in,bottom=1in,%
            footskip=.25in]{geometry}
\usepackage{blindtext}
\usepackage{titling}
\usepackage{amssymb}
\usepackage{amsmath}
\usepackage{listings}
\usepackage{lettrine} 
\usepackage{tikz}  
\usepackage{color} 
\usepackage{verbatim}
 \usetikzlibrary{shapes, arrows, calc, arrows.meta, fit, positioning} % these are the parameters passed to the library to create the node graphs  
\tikzset{  
    -Latex,auto,node distance =0.4 cm and 1.0 cm, thick,% node distance is the distance between one node to other, where 1.5cm is the length of the edge between the nodes  
    state/.style ={ellipse, draw, minimum width = 0.3 cm}, % the minimum width is the width of the ellipse, which is the size of the shape of vertex in the node graph  
    point/.style = {circle, draw, inner sep=0.18cm, fill, node contents={}},  
    bidirected/.style={Latex-Latex,dashed}, % it is the edge having two directions  
  treenode/.style = {align=center, inner sep=0pt, text centered,
    font=\sffamily},
    el/.style = {inner sep=2.5pt, align=right, sloped},
    arn_r/.style = {treenode, circle, red, draw=red},
    arn_x/.style = {treenode, rectangle, draw=black},
    minimum width=2.5em, minimum height=1.5em}% arbre rouge noir, nil
\setlength{\parskip}{12pt}
\title{Home Work 3 Undergraduate}
\date{\today}
\author{Jose Carlos Munoz}
%================================
\begin{document}
\newgeometry{left=0.8in,right=0.8in,top=1in,bottom=1in}
\begin{center}
    \Large
    \textbf{Homework 5 UG}\\
    \small
    \today\\
    \large
    Jose Carlos Munoz
\end{center}%===============================
FISHER LDA)\\

\begin{equation}
\begin{split}
X_{0} &= 
\begin{Bmatrix}
4 & 2.9 \\
3 & 6.4 \\
\end{Bmatrix}\\
X_{1} &= 
\begin{Bmatrix}
2 & 5.10 \\
2 & 2.15 \\
\end{Bmatrix}\\
\mu_0 &= 
\begin{Bmatrix}
3.5 & 4.65 \\
\end{Bmatrix}\\
\mu_1 &= 
\begin{Bmatrix}
2 & 3.625 \\
\end{Bmatrix}\\
X_{0}^{C} &= 
\begin{Bmatrix}
0.5 & -1.75 \\
0.5 & 1.75 \\
\end{Bmatrix}\\
X_{1}^{C} &= 
\begin{Bmatrix}
0 & 1.475 \\
0 & -1.475 \\
\end{Bmatrix}\\
\end{split} 
\end{equation}
-
\begin{equation}
\begin{split}
SC_0 &= X_{0}^{CT} * X_{0}^{C}\\
&= 
\begin{Bmatrix}
0.5 & -1.75 \\
-1.75 & 6.125 \\
\end{Bmatrix}\\
SC_1 &= X_{1}^{CT} * X_{1}^{C}\\
&= 
\begin{Bmatrix}
0 & 0 \\
0 & 4.35125 \\
\end{Bmatrix}\\
S_W &= SC_0 + SC_1\\
&= 
\begin{Bmatrix}
0.5 & -1.75 \\
-1.75 & 10.47625 \\
\end{Bmatrix}\\
S_B &= \{ \mu_0 - \mu_1\}^{T} * \{ \mu_0 - \mu_1\}\\
\{ \mu_0 - \mu_1\} &= 
\begin{Bmatrix}
3.5 & 4.65 \\
\end{Bmatrix}
 -
\begin{Bmatrix}
2 & 3.625 \\
\end{Bmatrix}\\
&= 
\begin{Bmatrix}
1 & 1.025 \\
\end{Bmatrix}\\
S_B &=
\begin{Bmatrix}
1        & 1.025 \\
1.025 & 1.050625 \\
\end{Bmatrix}\\
\end{split} 
\end{equation}
-
\begin{equation}
\begin{split}
S_{M}^{T} * S_{B} * \vec{w} & = \lambda * \vec{w}\\
S_{M}^{T} * S_{B} &= 
\begin{Bmatrix}
5.6398 & 5.7808 \\
1.0399 & 1.0654 \\
\end{Bmatrix}\\
\begin{Bmatrix}
5.6398 & 5.7808 \\
1.0399 & 1.0654 \\
\end{Bmatrix}
*\vec{w} &= \lambda * \vec{w}\\
\lambda_0 &= 6.70564\\
\lambda_1 &= 4.65e7\\
\vec{\lambda_0} &= 
\begin{Bmatrix}
0.983422 & 0.181330 \\
\end{Bmatrix}\\
\vec{w} &= \vec{\lambda_0}\\
\mu_{\vec{w}}^{0} &= \vec{w} * \mu_0\\
&= 4.2852\\
\mu_{\vec{w}}^{1} &= \vec{w} * \mu_1\\
&= 4.2852\\
sep &= \frac{\mu_{\vec{w}}^{1} + \mu_{\vec{w}}^{2}}{2}
sep = 3.4547
\end{split} 
\end{equation}
To classify point (3.8,5) we multiply it by $\vec{w}$ and see if it is less than  or greater than the seperator. If it is less than the seperator then we can conclude that its Class 1 otherwise it is class 0.
\begin{equation}
\begin{split}
val &= \{3.8,5\} * \vec{w}\\
val &= 3.87243
\end{split}
\end{equation}
the value is greater than the seperator so we classify the value as Class 0\par
PERCEPTRON\\
The data is as presented\\
\begin{equation}
\begin{array}{c|ccc|c}
 & X_1 &  X_2  &  X_3 & Y\\
\hline
a &4 &  3  &  6 & -1\\
\hline
b& 2 &  -2  & 3 & 1\\
\hline
c & 1 & 0  & -3 & 1\\
\hline
d& 4 & 2  & 3 & -1\\
\end{array}
\end{equation}

\end{document}
