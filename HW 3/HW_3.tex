\documentclass[12pt,english]{article}
\usepackage[a4paper,bindingoffset=0.2in,%
            left=1in,right=1in,top=1in,bottom=1in,%
            footskip=.25in]{geometry}
\usepackage{blindtext}
\usepackage{titling}
\usepackage{amssymb}
\usepackage{amsmath}
\usepackage{listings}
\usepackage{lettrine} 
\usepackage{tikz}  
\usepackage{color} 
\usepackage{verbatim}
 \usetikzlibrary{shapes, arrows, calc, arrows.meta, fit, positioning} % these are the parameters passed to the library to create the node graphs  
\tikzset{  
    -Latex,auto,node distance =0.4 cm and 1.0 cm, thick,% node distance is the distance between one node to other, where 1.5cm is the length of the edge between the nodes  
    state/.style ={ellipse, draw, minimum width = 0.3 cm}, % the minimum width is the width of the ellipse, which is the size of the shape of vertex in the node graph  
    point/.style = {circle, draw, inner sep=0.18cm, fill, node contents={}},  
    bidirected/.style={Latex-Latex,dashed}, % it is the edge having two directions  
  treenode/.style = {align=center, inner sep=0pt, text centered,
    font=\sffamily},
    el/.style = {inner sep=2.5pt, align=right, sloped},
    arn_r/.style = {treenode, circle, red, draw=red},
    arn_x/.style = {treenode, rectangle, draw=black},
    minimum width=2.5em, minimum height=1.5em}% arbre rouge noir, nil
\setlength{\parskip}{12pt}
\title{Home Work 3 Undergraduate}
\date{\today}
\author{Jose Carlos Munoz}
%================================
\begin{document}
\newgeometry{left=0.8in,right=0.8in,top=1in,bottom=1in}
\begin{center}
    \Large
    \textbf{Homework 3}\\
    \small
    \today\\
    \large
    Jose Carlos Munoz
\end{center}%===============================
\par
Problem 1)\\
The First thing we have to do is to Create 4 differnet contingency tables for the 4 different possible splits
\begin{equation}\tag{1}\label{eq:1}
\begin{split}
\mbox{Age} &= 
\begin{array}{c|ccc|c}
\mbox{Age}   & \mbox{None} & \mbox{Soft} & \mbox{Hard} & \mbox{Total}\\
\hline
\mbox{Young} & 4           & 2           & 2           &  8\\
\mbox{Pre}   & 5           & 2           & 1           &  8\\
\mbox{Pres}  & 6           & 1           & 1           &  8\\
\hline
\Sigma       & 15          & 5           & 4           & 24\\
\end{array}
\\
\mbox{Spectable Prescription} &= 
\begin{array}{c|ccc|c}
\mbox{Spect} & \mbox{None} & \mbox{Soft} & \mbox{Hard} & \\
\hline
\mbox{M}     & 7           & 2           & 3           & 12\\
\mbox{H}     & 8           & 3           & 1           & 12\\
\hline
\Sigma       & 15          & 5           & 4           & 24\\
\end{array}
\\
\mbox{Astigmatism } &=
\begin{array}{c|ccc|c}
\mbox{Ast}   & \mbox{None} & \mbox{Soft} & \mbox{Hard} & \\
\hline
\mbox{Y}     & 8           & 0           & 4           & 12\\
\mbox{N}     & 4           & 5           & 0           & 12\\
\hline
\Sigma       & 15          & 5           & 4           & 24\\
\end{array}
\\
\mbox{Tear Production Rate} &=
\begin{array}{c|ccc|c}
\mbox{TRB}   & \mbox{None} & \mbox{Soft} & \mbox{Hard} & \\
\hline
\mbox{N}     & 3           & 5           & 4           & 12\\
\mbox{R}     & 12          & 0           & 0           & 12\\
\hline
\Sigma       & 15          & 5           & 4           & 24\\
\end{array}
\end{split}
\end{equation}
From these table we are now going to calculate the expected Table if we assume that the events are independent

\begin{equation}\tag{2}\label{eq:2}
\begin{split}
\mbox{Age}_{expected} &= 
\begin{array}{c|ccc}
\mbox{Age}   & \mbox{None} & \mbox{Soft} & \mbox{Hard}\\
\hline
\mbox{Young} & 5 & \frac{5}{3} & \frac{4}{3}\\
\mbox{Pre}   & 5 & \frac{5}{3} & \frac{4}{3}\\
\mbox{Pres}  & 5 & \frac{5}{3} & \frac{4}{3}\\
\end{array}
\\
\mbox{Spectable Prescription}_{expected} &= 
\begin{array}{c|ccc}
\mbox{Spect} & \mbox{None} & \mbox{Soft} & \mbox{Hard}\\
\hline
\mbox{M}     & 7.5 & 2.5 & 2 \\
\mbox{H}     & 7.5 & 2.5 & 2 \\
\end{array}
\\
\mbox{Astigmatism }_{expected} &=
\begin{array}{c|ccc}
\mbox{Ast}   & \mbox{None} & \mbox{Soft} & \mbox{Hard}\\
\hline
\mbox{Y}     & 7.5 & 2.5 & 2 \\
\mbox{N}     & 7.5 & 2.5 & 2 \\
\end{array}
\\
\mbox{Tear Production Rate}_{expected} &=
\begin{array}{c|ccc}
\mbox{TRB}   & \mbox{None} & \mbox{Soft} & \mbox{Hard}\\
\hline
\mbox{N}     & 7.5 & 2.5 & 2 \\
\mbox{R}     & 7.5 & 2.5 & 2 \\
\end{array}
\end{split}
\end{equation}
Now that we have both observed and expected tables, we can now find the $\mathcal{X}^2$. To find it, we must first calculate each of the corresponding observed and expected value, $\frac{(observed - Expected)^2}{Expected}$
\begin{equation}\tag{3}\label{eq:3}
\begin{split}
\mbox{Age} &=
\begin{array}{|ccc|}
\frac{1}{5} & \frac{1}{5}  & \frac{1}{3} \\
\hline
0           & \frac{1}{15} & \frac{1}{12} \\
\hline
\frac{1}{5} & \frac{4}{15} & \frac{1}{12} \\
\end{array}
\\
\mbox{Spectable Prescription} &=
\begin{array}{|ccc|}
\frac{1}{30} & \frac{1}{10} & \frac{1}{2} \\
\hline
\frac{1}{30} & \frac{1}{10} & \frac{1}{2} \\
\end{array}
\\
\mbox{Astigmatism} &=
\begin{array}{|ccc|}
\frac{1}{30} & 2\frac{1}{2} & 2 \\
\hline
\frac{1}{30} & 2\frac{1}{2} & 2 \\
\end{array}
\\
\mbox{Tear Production Rate} &=
\begin{array}{|ccc|}
2\frac{7}{10} & 2\frac{1}{2} & 2 \\
\hline
2\frac{7}{10} & 2\frac{1}{2} & 2 \\
\end{array}
\end{split}
\end{equation}
Once all values are found, we add them all up and get our $\mathcal{X}^2$.
\begin{equation}\tag{4}\label{eq:4}
\begin{split}
\mathcal{X}^{2}_{age} &= 3\frac{2}{15}\\
\mathcal{X}^{2}_{SpPr} &= 1\frac{8}{30}\\
\mathcal{X}^{2}_{As} &= 9\frac{2}{30}\\
\mathcal{X}^{2}_{TRB} &= 14\frac{4}{10}\\
\end{split}
\end{equation}
Once we founde these values we find the degrees of freedom for alll of them and see what is th corresponding critical $\mathcal{X}^2$ value. Age is the only one with a different degree of freedom , the other attributes have the same degree of freedom. For age, the Degree of freedom is 4, the rest have  a degree of freedom of 2. So for Age its Critical $\mathcal{X}^2$ value is 9.488 and the rest are 5.991.\\
We can now say that we are able to split this node because of all the $\mathcal{X}^2$ values, 2 are higher than the critical value. Which means we can split from either of these. We split in the Tear Production attribute becuse it has the highest $\mathcal{X}^2$ value from either options.
\par
Problem 2\\
The Tree is over fitted because at node n=1,s=5, h= 0, it is splitting to a complex  structure with a very simple sample size.\\
To determine if we need to see if that node needs to be splitted, we would have to see attributes in there along with the observes results there.
\begin{equation}\tag{1}\label{eq:5}
\begin{split}
\mbox{Node TPR}_{Normal} &=
\begin{array}{c|c|c|c}
 \mbox{Age}  & \mbox{SP}& \mbox{A} & \mbox{RCL}\\
 \hline
 \mbox{Y}    & \mbox{M} & \mbox{N} & \mbox{S} \\
 \mbox{Y}    & \mbox{M} & \mbox{Y} & \mbox{H} \\
 \mbox{Y}    & \mbox{H} & \mbox{N} & \mbox{S} \\
 \mbox{Y}    & \mbox{H} & \mbox{Y} & \mbox{H} \\
 \mbox{Pre}  & \mbox{M} & \mbox{N} & \mbox{S} \\
 \mbox{Pre}  & \mbox{M} & \mbox{Y} & \mbox{H} \\
 \mbox{Pre}  & \mbox{H} & \mbox{N} & \mbox{S} \\
 \mbox{Pre}  & \mbox{H} & \mbox{Y} & \mbox{N} \\
 \mbox{Pres} & \mbox{M} & \mbox{N} & \mbox{N} \\
 \mbox{Pres} & \mbox{M} & \mbox{Y} & \mbox{H} \\
 \mbox{Pres} & \mbox{H} & \mbox{N} & \mbox{S} \\
 \mbox{Pres} & \mbox{H} & \mbox{Y} & \mbox{N} 
\end{array}
\\
\\
\mbox{Node A}_{No} &=
\begin{array}{c|c|c}
 \mbox{Age}  & \mbox{SP}& \mbox{RCL}\\
 \hline
 \mbox{Y}    & \mbox{M} & \mbox{S} \\
 \mbox{Y}    & \mbox{H} & \mbox{S} \\
 \mbox{Pre}  & \mbox{M} & \mbox{S} \\
 \mbox{Pre}  & \mbox{H} & \mbox{S} \\
 \mbox{Pres} & \mbox{M} & \mbox{N} \\
 \mbox{Pres} & \mbox{H} & \mbox{S} 
\end{array}
\end{split}
\end{equation}
From this Node we create a contigency table to See if it is worth splitting for either Attributes
\begin{equation}\tag{2}\label{eq:6}
\begin{split}
\mbox{Age} &= 
\begin{array}{c|ccc|c}
\mbox{Age}   & \mbox{None} & \mbox{Soft} & \mbox{Hard} & \\
\hline
\mbox{Young} & 0 & 2 & 0 & 2 \\
\mbox{Pre}     & 0 & 2 & 0 & 2\\
\mbox{Pres}   & 1 & 1 & 0 & 2\\
\hline
\Sigma           & 1 & 5 & 0 & 6
\end{array}
\\
\\
\mbox{SP} &= 
\begin{array}{c|ccc|c}
\mbox{SP}   & \mbox{None} & \mbox{Soft} & \mbox{Hard} & \\
\hline
\mbox{M} & 1 & 2 & 0 & 3 \\
\mbox{H} & 0 & 3 & 0 & 3\\
\hline
\Sigma     & 1 & 5 & 0 & 6
\end{array}
\end{split}
\end{equation}
From both tables we can remove the last column as we can not physically split into that node.
so our table now looks like this
\begin{equation}\tag{2}\label{eq:6}
\begin{split}
\mbox{Age} &= 
\begin{array}{c|cc|c}
\mbox{Age}   & \mbox{None} & \mbox{Soft} & \\
\hline
\mbox{Young} & 0 & 2 & 2 \\
\mbox{Pre}     & 0 & 2 & 2\\
\mbox{Pres}   & 1 & 1 & 2\\
\hline
\Sigma           & 1 & 5 & 6
\end{array}
\\
\\
\mbox{SP} &= 
\begin{array}{c|cc|c}
\mbox{SP}   & \mbox{None} & \mbox{Soft}& \\
\hline
\mbox{M} & 1 & 2 & 3 \\
\mbox{H} & 0 & 3 & 3\\
\hline
\Sigma     & 1 & 5 & 6
\end{array}
\end{split}
\end{equation}
We then calculate the Expected table to see what the results we expect are
\begin{equation}\tag{3}\label{eq:7}
\begin{split}
\mbox{Age} &= 
\begin{array}{|cc|}
 \mbox{None} & \mbox{Soft}\\
\hline
 \frac{1}{3} & \frac{5}{3}\\
\hline
 \frac{1}{3} & \frac{5}{3}\\
\hline
 \frac{1}{3} & \frac{5}{3}\\
\hline
\end{array}
\\
\\
\mbox{SP} &= 
\begin{array}{|cc|}
 \mbox{None} & \mbox{Soft}\\
\hline
 \frac{1}{2} & 2\frac{1}{2}\\
\hline
 \frac{1}{2} & 2\frac{1}{2}\\
\hline
\end{array}
\end{split}
\end{equation}
Now we can calculate the $\mathcal{X}^2$

\begin{equation}\tag{3}\label{eq:7}
\begin{split}
\mbox{Age} &= 
\begin{array}{|cc|}
 \mbox{None} & \mbox{Soft}\\
\hline
1\frac{1}{3} & \frac{1}{15}\\
\hline
 1\frac{1}{3} & \frac{1}{15}\\
\hline
 \frac{1}{3} & \frac{4}{15}\\
\hline
\end{array}
\\
\mathcal{X}^2 &= 3.4\\
\\
\\
\mbox{SP} &= 
\begin{array}{|cc|}
 \mbox{None} & \mbox{Soft}\\
\hline
 \frac{1}{2} & 2\frac{1}{10}\\
\hline
 \frac{1}{2} & 2\frac{1}{10}\\
\hline
\end{array}
\\
\mathcal{X}^2 &=1.2
\end{split}
\end{equation}
The degree of freedom for the age is 2 and the degree of freedom for the SP is 1. Their critical $\mathcal{X}^2$ respectivly with a 0.05 confidence level is 5.991 and 3.841 respectively. From this, we can conclude that there is no point of splitting this node because their $\mathcal{X}^2$ value is well below their respective critical value. Therefore making this split unnecessary.
\end{document}
