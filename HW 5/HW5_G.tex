\documentclass[12pt,english]{article}
\usepackage[a4paper,bindingoffset=0.2in,%
            left=1in,right=1in,top=1in,bottom=1in,%
            footskip=.25in]{geometry}
\usepackage{blindtext}
\usepackage{titling}
\usepackage{amssymb}
\usepackage{amsmath}
\usepackage{listings}
\usepackage{lettrine} 
\usepackage{tikz}  
\usepackage{color} 
\usepackage{verbatim}
 \usetikzlibrary{shapes, arrows, calc, arrows.meta, fit, positioning} % these are the parameters passed to the library to create the node graphs  
\tikzset{  
    -Latex,auto,node distance =0.4 cm and 1.0 cm, thick,% node distance is the distance between one node to other, where 1.5cm is the length of the edge between the nodes  
    state/.style ={ellipse, draw, minimum width = 0.3 cm}, % the minimum width is the width of the ellipse, which is the size of the shape of vertex in the node graph  
    point/.style = {circle, draw, inner sep=0.18cm, fill, node contents={}},  
    bidirected/.style={Latex-Latex,dashed}, % it is the edge having two directions  
  treenode/.style = {align=center, inner sep=0pt, text centered,
    font=\sffamily},
    el/.style = {inner sep=2.5pt, align=right, sloped},
    arn_r/.style = {treenode, circle, red, draw=red},
    arn_x/.style = {treenode, rectangle, draw=black},
    minimum width=2.5em, minimum height=1.5em}% arbre rouge noir, nil
\setlength{\parskip}{12pt}
\title{Home Work 5 Graduate}
\date{\today}
\author{Jose Carlos Munoz}
%================================
\begin{document}
\newgeometry{left=0.8in,right=0.8in,top=1in,bottom=1in}
\begin{center}
    \Large
    \textbf{Homework 5 G}\\
    \small
    \today\\
    \large
    Jose Carlos Munoz
\end{center}%===============================
\par
12.a)\\
\par
1-nearest for 5.0 are $4.9 \vert +$
\begin{equation*}
\begin{array}{c|cc}
              & + & - \\
\hline          
\mbox{Amount} & 1 & 0 \\
\end{array}
\end{equation*}
The maximum amount is for +. So it is classified as a +\\
3-nearest for 5.0 are $4.9 \vert +$, $5.2 \vert -$, $5.3 \vert -$
\begin{equation*}
\begin{array}{c|cc}
              & + & - \\
\hline          
\mbox{Amount} & 1 & 2 \\
\end{array}
\end{equation*}
The maximum amount is for -. So it is classified as a -\\
5-nearest for 5.0 are $4.9 \vert +$, $5.2 \vert -$, $5.3 \vert -$,$5.5 \vert +$, $4.6 \vert +$
\begin{equation*}
\begin{array}{c|cc}
              & + & - \\
\hline          
\mbox{Amount} & 3 & 2 \\
\end{array}
\end{equation*}
The maximum amount is for +. So it is classified as a +\\
9-nearest for 5.0 are $4.9 \vert +$, $5.2 \vert -$, $5.3 \vert -$,$5.5 \vert +$, $4.6 \vert +$,$4.5 \vert +$, $3.0 \vert -$, $7.0 \vert -$,$0.5 \vert -$
\begin{equation*}
\begin{array}{c|cc}
              & + & - \\
\hline          
\mbox{Amount} & 4 & 5 \\
\end{array}
\end{equation*}
The maximum amount is for -. So it is classified as a -\\
12.b)\\
1-nearest for 5.0 are $4.9 \vert +$
\begin{equation*}
\begin{split}
\Sigma_+ &= ( 5 -4.9 )^{-2}\\
&= 100\\
\Sigma_- &= 0
\end{split}
\end{equation*}
The maximum value is for +. So it is classified as a +\\
3-nearest for 5.0 are $4.9 \vert +$, $5.2 \vert -$, $5.3 \vert -$
\begin{equation*}
\begin{split}
\Sigma_+ &= ( 5 - 4.9 )^{-2)}\\
&= 100\\
\Sigma_- &= ( 5 - 5.2 )^{-2} +( 5 - 5.3 )^{-2}\\
&= 36.11\\
\end{split}
\end{equation*}
The maximum amount is for +. So it is classified as a +\\
5-nearest for 5.0 are $4.9 \vert +$, $5.2 \vert -$, $5.3 \vert -$,$5.5 \vert +$, $4.6 \vert +$
\begin{equation*}
\begin{split}
\Sigma_+ &= ( 5 - 4.9 )^{-2} +( 5 - 5.5 )^{-2} +( 5 - 4.6 )^{-2}\\
&= 110.25\\
\Sigma_- &= ( 5 - 5.2 )^{-2} +( 5 - 5.3 )^{-2}\\
&= 36.11\\
\end{split}
\end{equation*}
The maximum amount is for +. So it is classified as a +\\
9-nearest for 5.0 are $4.9 \vert +$, $5.2 \vert -$, $5.3 \vert -$,$5.5 \vert +$, $4.6 \vert +$,$4.5 \vert +$, $3.0 \vert -$, $7.0 \vert -$,$0.5 \vert -$
\begin{equation*}
\begin{split}
\Sigma_+ &= ( 5 - 4.9 )^{-2} +( 5 - 5.5 )^{-2} +( 5 - 4.6 )^{-2}+( 5 - 4.5 )^{-2}\\
&= 114.25\\
\Sigma_- &= ( 5 - 5.2 )^{-2} +( 5 - 5.3 )^{-2}+( 5 - 3.0 )^{-2}+( 5 - 7.0 )^{-2}+( 5 - 0.5 )^{-2}\\
&= 36.7\\
\end{split}
\end{equation*}
The maximum amount is for +. So it is classified as a +
\par
13)\\
We first make a table for the Home Owener and Marraige Attributes
\begin{equation}\tag{Marraige Status}
\begin{array}{c|ccc}
\mbox{Class}& S & M & D \\
\hline          
\mbox{Yes}   & 2 & 0 & 1 \\
\mbox{No}    & 2 & 4 & 1 \\
\end{array}
\end{equation}

\begin{equation}\tag{Home Owner}
\begin{array}{c|ccc}
\mbox{Class}& Y & N  \\
\hline          
\mbox{Yes}   & 0 & 3 \\
\mbox{No}    & 3 & 4 \\
\end{array}
\end{equation}
we use this equation to find the distance from each of the values
\begin{equation}
D(V_1,V_2) = \Sigma \vert \frac{n_{i1}}{n_1} - \frac{n_{i2}}{n_2}\vert
\end{equation}
We First do the Marital Status Attribute
\begin{equation*}
\begin{split}
d(\mbox{Single},\mbox{Married}) &= \vert\frac{2}{4} - \frac{0}{4}\vert + \vert\frac{2}{4} - \frac{4}{4}\vert\\
&= \frac{1}{2} + \frac{1}{2}\\
&= 1\\
d(\mbox{Single},\mbox{Divorced}) &= \vert\frac{2}{4} - \frac{1}{2}\vert + \vert\frac{2}{4} - \frac{1}{2}\vert\\
&= 0 + 0\\
&= 1\\
d(\mbox{Divorced},\mbox{Married}) &= \vert\frac{1}{2} - \frac{0}{4}\vert + \vert\frac{1}{2} - \frac{4}{4}\vert\\
&= \frac{1}{2} + \frac{1}{2}\\
&= 1\\
\end{split}
\end{equation*}
Then we do for the Home Owner Attribute
\begin{equation*}
\begin{split}
d(\mbox{Yes},\mbox{No}) &= \vert\frac{0}{3} - \frac{3}{7}\vert + \vert\frac{3}{3} - \frac{4}{7}\vert\\
&= \frac{3}{7} + \frac{3}{7}\\
&= \frac{6}{7}\\
\end{split}
\end{equation*}

\end{document}
