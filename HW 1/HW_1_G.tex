\documentclass[12pt,english]{article}
\usepackage[a4paper,bindingoffset=0.2in,%
            left=1in,right=1in,top=1in,bottom=1in,%
            footskip=.25in]{geometry}
\usepackage{blindtext}
\usepackage{titling}
\usepackage{amssymb}
\usepackage{amsmath}
\usepackage{listings}
\usepackage{lettrine} 
\usepackage{tikz}  
\usepackage{color} 
 \usetikzlibrary{shapes, arrows, calc, arrows.meta, fit, positioning} % these are the parameters passed to the library to create the node graphs  
\tikzset{  
    -Latex,auto,node distance =0.6 cm and 1.3 cm, thick,% node distance is the distance between one node to other, where 1.5cm is the length of the edge between the nodes  
    state/.style ={ellipse, draw, minimum width = 0.9 cm}, % the minimum width is the width of the ellipse, which is the size of the shape of vertex in the node graph  
    point/.style = {circle, draw, inner sep=0.18cm, fill, node contents={}},  
    bidirected/.style={Latex-Latex,dashed}, % it is the edge having two directions  
    el/.style = {inner sep=2.5pt, align=right, sloped}  
}  
\setlength{\parskip}{12pt}
\title{Home Work 1 Graduate}
\date{\today}
\author{Jose Carlos Munoz}
%================================
\begin{document}
\newgeometry{left=0.8in,right=0.8in,top=1in,bottom=1in}
\begin{center}
    \Large
    \textbf{Homework 1 Graduate}\\
    \small
    \today\\
    \large
    Jose Carlos Munoz
\end{center}
Question 1)\par
For time series clustering, the positive correlation are what matters the most. This is because we want to cluster items that are correlated to each other the most. So any negative correlation is not taken into account.So for this situation it would be wise that clamp the values from 0 to 1 .As shown in equation 1.
\begin{equation}
transform =
\begin{Bmatrix} corr &if &corr \geq 0  \\  0     & if & corr < 0  \\  \end{Bmatrix}
\end{equation}
For predicting a behavior change between one time series for another, what we are after is just if there is a certain correlation;regardless if it is positive or negative. So for this transform, it would just be the absolute value of the correlation value, or as $transform =|corr|$.
\\
\\
Question 2)\par
It is well known that in nature that most things that grow are symetrical. So as a the width grows sow does the length of a petal. Since this is usually the case in nature, there should be a large correlation within that bucket. The reason that most points fall in the buckets is that it further cements that there is a correlation of petal width and petal length as a plant grows.
%===============================
\end{document}
