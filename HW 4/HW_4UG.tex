\documentclass[12pt,english]{article}
\usepackage[a4paper,bindingoffset=0.2in,%
            left=1in,right=1in,top=1in,bottom=1in,%
            footskip=.25in]{geometry}
\usepackage{blindtext}
\usepackage{titling}
\usepackage{amssymb}
\usepackage{amsmath}
\usepackage{listings}
\usepackage{lettrine} 
\usepackage{tikz}  
\usepackage{color} 
\usepackage{verbatim}
 \usetikzlibrary{shapes, arrows, calc, arrows.meta, fit, positioning} % these are the parameters passed to the library to create the node graphs  
\tikzset{  
    -Latex,auto,node distance =0.4 cm and 1.0 cm, thick,% node distance is the distance between one node to other, where 1.5cm is the length of the edge between the nodes  
    state/.style ={ellipse, draw, minimum width = 0.3 cm}, % the minimum width is the width of the ellipse, which is the size of the shape of vertex in the node graph  
    point/.style = {circle, draw, inner sep=0.18cm, fill, node contents={}},  
    bidirected/.style={Latex-Latex,dashed}, % it is the edge having two directions  
  treenode/.style = {align=center, inner sep=0pt, text centered,
    font=\sffamily},
    el/.style = {inner sep=2.5pt, align=right, sloped},
    arn_r/.style = {treenode, circle, red, draw=red},
    arn_x/.style = {treenode, rectangle, draw=black},
    minimum width=2.5em, minimum height=1.5em}% arbre rouge noir, nil
\setlength{\parskip}{12pt}
\title{Home Work 3 Undergraduate}
\date{\today}
\author{Jose Carlos Munoz}
%================================
\begin{document}
\newgeometry{left=0.8in,right=0.8in,top=1in,bottom=1in}
\begin{center}
    \Large
    \textbf{Homework 4 UG}\\
    \small
    \today\\
    \large
    Jose Carlos Munoz
\end{center}%===============================
4.6)a)\\
\begin{equation}\tag{1}\label{eq:1}
\begin{split}
P(\mbox{S}\vert \mbox{UG}) &= .15 \\
P(S\vert G) &= .23\\
P(G) &= .2\\
P(UG) &= .8
\end{split}
\end{equation}
These are the known probabilites. From this we can find $P(\mbox{G}\vert \mbox{S})$.\\
Because of Bayes Theroem  $P(\mbox{G}\vert \mbox{S})$ is the same as the following
\begin{equation}
P(\mbox{G}\vert \mbox{S}) = \frac{P(\mbox{S} \vert \mbox{G}) * P(\mbox{G})}{P(\mbox{S})}
\end{equation}
$P(\mbox{S})$ can be found as
\begin{equation}
\begin{split}
P(\mbox{S}) &=  P(\mbox{S} \vert \mbox{G}) * P(\mbox{G}) + P(\mbox{S} \vert \mbox{UG}) * P(\mbox{UG})\\
P(\mbox{S}) &= .23 *.2 + .15 * .8\\
P(\mbox{S}) &=.166
\end{split}
\end{equation}
Therefore
\begin{equation}
\begin{split}
P(\mbox{G}\vert \mbox{S}) &= \frac{.23 * .2}{.166}\\
P(\mbox{G}\vert \mbox{S}) &= .277\\
\end{split}
\end{equation}

\end{document}
